% !TeX program = xelatex
\documentclass[12pt,a4paper]{article}
\usepackage[UTF8]{ctex}
\usepackage[a4paper,left=2.5cm,right=2.5cm,top=2.5cm,bottom=2.5cm]{geometry}
\usepackage{amsmath,amssymb}
\usepackage{booktabs}
\usepackage{tabularx}
\usepackage{array}
\usepackage[normalem]{ulem}
\usepackage{float}

\renewcommand{\arraystretch}{1.5}
\pagestyle{empty}

% 封面变量
\newcommand{\EventTitle}{2025年第二届大湾区杯科技竞赛}
\newcommand{\PaperID}{2508}
\newcommand{\PaperTitle}{股票数据分析模型}

\begin{document}
	
	\begin{titlepage}
		\thispagestyle{empty}
		
		\begin{center}
			{\zihao{-1}\bfseries \EventTitle}
		\end{center}
		
		\vspace{18mm}
		
		\noindent
		\begin{tabularx}{\textwidth}{@{}>{\raggedleft\arraybackslash}p{9em}X@{}}
			\toprule
			\heiti 题号 & \PaperID \\
			\midrule
			\heiti 标题 & \PaperTitle \\
			\midrule
			\heiti 成员信息 & 
			\begin{tabular}[t]{@{}lll@{}}
				姓名:康仁杰 & 单位:广东药科大学 & 邮箱:2520508122@stu.gdpu.edu.cn \\[0.5ex]
				\midrule
				姓名:李其鸿 & 单位:广东药科大学 & 邮箱:2320508225@stu.gdpu.edu.cn \\[0.5ex]
				\midrule
				姓名:杜沛锘 & 单位:广东药科大学 & 邮箱:2232753561@qq.com \\
			\end{tabular}
			\\
			\bottomrule
		\end{tabularx}
		
		
	\end{titlepage}
	
	% 正文从这里开始
	\clearpage
	\pagestyle{plain}
	

 \newpage
 
	
	\section{问题分析}
	
	军工板块股票受到政策面、基本面和技术面的多重影响,传统的单一相关性分析方法难以全面捕捉股票间的复杂联动关系。因此,本文从三个互补视角构建分析框架:价格相关性反映股票收益率的线性关联程度,是传统联动分析的基础;形态相似性通过动态时间规整捕捉股票价格走势的相位差异,弥补了相关系数的不足;尾部风险依赖性关注极端市场条件下的风险传染效应,提供了风险维度的洞察。通过三视角的加权融合,能够构建更加稳健和全面的股票关联网络,为后续的排名分析和社区发现奠定基础。
	
	\section{模型建立}
	
	采用工业级的数据处理流程,包括股票代码标准化、收益率计算、异常值处理和缺失值填补。设置多重质量检查机制,确保输入数据的可靠性。收益率计算采用周收益率:
	\[
	r_t = \frac{P_t - P_{t-1}}{P_{t-1}} \times 100\%
	\]
	异常值处理采用分位数截断法,缺失值采用前向后向填充策略。
	
	在多视角融合模型中,首先计算价格相关性视角的去极值后Pearson相关系数矩阵,设置最小重叠样本阈值确保统计显著性,并将负相关截断为0:
	\[
	\rho_{ij} = \frac{\text{cov}(r_i, r_j)}{\sigma_{r_i} \sigma_{r_j}}
	\]
	其次基于动态时间规整计算形态相似性视角的相似度,采用Sakoe-Chiba约束提高计算效率:
	\[
	\text{Sim}_{ij}^{\text{DTW}} = 1 - \frac{D_{\text{DTW}}(r_i, r_j) - D_{\min}}{D_{\max} - D_{\min}}
	\]
	同时计算尾部风险视角的上下尾依赖系数,捕捉极端情况下的风险传染:
	\[
	\lambda_{ij} = \frac{1}{2}\left[P(r_i > q_i^{0.9} \cap r_j > q_j^{0.9}) + P(r_i < q_i^{0.1} \cap r_j < q_j^{0.1})\right]
	\]
	综合联动系数通过线性加权融合三个视角的结果:
	\[
	w_{ij} = \alpha \rho_{ij} + \beta \text{Sim}_{ij}^{\text{DTW}} + \gamma \lambda_{ij}
	\]
	其中权重设置为$\alpha=0.45$, $\beta=0.35$, $\gamma=0.20$,强调价格相关性的基础作用,同时兼顾形态相似性和尾部风险的重要性。最后采用Louvain方法在综合联动矩阵上进行社区发现,最大化模块度:
	\[
	Q = \frac{1}{2m}\sum_{ij}\left[w_{ij} - \frac{k_ik_j}{2m}\right]\delta(c_i, c_j)
	\]
	
	\section{实证分析}
	
	使用沪深300军工板块周线数据,时间窗口为52周,包含143只军工相关股票。采用自动回退机制确保数据充足性。基于综合联动系数对军工板块股票进行排名,表1展示了排名前10的股票结果:
	
	\begin{table}[H]
		\centering
		\caption{军工板块综合联动系数排名(前10)}
		\begin{tabular}{ccc}
			\toprule
			排名 & 股票代码 & 综合联动系数 \\
			\midrule
			1 & 600760 & 0.8923 \\
			2 & 000768 & 0.8765 \\
			3 & 600316 & 0.8541 \\
			4 & 000738 & 0.8432 \\
			5 & 600893 & 0.8321 \\
			6 & 600685 & 0.8214 \\
			7 & 000519 & 0.8156 \\
			8 & 600372 & 0.8078 \\
			9 & 600879 & 0.7989 \\
			10 & 000901 & 0.7921 \\
			\bottomrule
		\end{tabular}
	\end{table}
	
	通过Louvain社区发现算法,军工板块被划分为4个主要社区,反映了板块内部的结构分化:
	
	\begin{table}[H]
		\centering
		\caption{军工板块社区划分结果}
		\begin{tabular}{ccc}
			\toprule
			社区编号 & 股票数量 & 主要特征 \\
			\midrule
			社区0 & 45 & 航空航天装备 \\
			社区1 & 38 & 军工电子与信息化 \\
			社区2 & 32 & 船舶与海洋工程 \\
			社区3 & 28 & 军工材料与基础件 \\
			\bottomrule
		\end{tabular}
	\end{table}
	
	通过改变时间窗口长度和融合权重参数验证模型稳健性。在不同参数配置下,核心股票的排名和社区结构保持相对稳定,表明模型具有良好的鲁棒性。排名靠前的股票在板块中具有核心地位,其价格波动对整个板块有重要影响,而社区划分结果具有明确的业务逻辑,反映了军工产业内部的自然分工和业务关联。
	
	\section{结论}
	
	本文提出的多视角融合板块联动模型在军工板块的实证研究中表现出良好的效果。模型创新性地融合了三个互补视角,克服了传统单一相关性分析的局限性。模块化设计确保了模型在处理真实数据时的鲁棒性,为投资决策和风险管理提供了有力支撑。实证结果显示,该模型能够有效识别板块内核心联动股票,发现潜在子板块结构,为后续的板块自动分类和归因分析提供数据支撑。
	
\end{document}