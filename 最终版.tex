% !TeX program = xelatex
\documentclass[12pt,a4paper]{article}

% 中文支持
\usepackage[UTF8]{ctex}

% 版式与常用宏包
\usepackage[a4paper,left=2.5cm,right=2.5cm,top=2.5cm,bottom=2.5cm]{geometry}
\usepackage{amsmath,amssymb}
\usepackage{booktabs,tabularx,array}
\usepackage[normalem]{ulem}
\usepackage{graphicx}
\usepackage{xparse}     % 稳健的命令定义(可选参数)
\usepackage{caption}
\usepackage{xcolor}
\usepackage{listings}

% 超链接尽量最后加载
\usepackage{hyperref}
\hypersetup{
	colorlinks=true,
	linkcolor=blue!60!black,
	urlcolor=blue!60!black,
	citecolor=blue!60!black
}

\renewcommand{\arraystretch}{1.5}

% 安全的插图宏:若图片不存在,给出占位提示而不报错
% 用法:\IncludeFigure[width=0.9\linewidth]{path/to/img.png}
\NewDocumentCommand{\IncludeFigure}{O{} m}{%
	\IfFileExists{#2}{%
		\includegraphics[#1]{#2}%
	}{%
		\fbox{\parbox[c][0.25\textheight][c]{0.9\linewidth}{\centering 缺少图片:\texttt{#2}\\请先生成该图片后再编译}}}%
}%

% 封面变量
\newcommand{\EventTitle}{2025年第二届大湾区杯科技竞赛}
\newcommand{\PaperID}{2508}
\newcommand{\PaperTitle}{股票数据分析模型}

\pagestyle{empty}

\begin{document}

% 封面
\begin{titlepage}
	\thispagestyle{empty}
	
	\begin{center}
		{\zihao{-1}\bfseries \EventTitle}
	\end{center}
	
	\vspace{18mm}
	
	\noindent
	\begin{tabularx}{\textwidth}{@{}>{\raggedleft\arraybackslash}p{9em}X@{}}
		\toprule
		\heiti 题号 & \PaperID \\
		\midrule
		\heiti 标题 & \PaperTitle \\
		\midrule
		\heiti 成员信息 &
		\begin{tabular}[t]{@{}lll@{}}
			姓名:康仁杰 & 单位:广东药科大学 & 邮箱:2520508122@stu.gdpu.edu.cn \\[0.5ex]
			\midrule
			姓名:李其鸿 & 单位:广东药科大学 & 邮箱:2320508225@stu.gdpu.edu.cn \\[0.5ex]
			\midrule
			姓名:杜沛锘 & 单位:广东药科大学 & 邮箱:2232753561@qq.com \\
		\end{tabular}
		\\
		\bottomrule
	\end{tabularx}
	
	\vfill
	
	{\large
		\noindent 签名(可电子签名):
		\quad 康仁杰、李其鸿、杜沛锘
	}
	
\end{titlepage}

% 正文
\clearpage
\pagestyle{plain}

\section{问题1:板块联动建模与排名(军工板块)}
\label{sec:problem1}

\subsection{任务与数据设定}
\label{subsec:task-data}
本文聚焦题目一:基于交易数据时间序列构建军工板块的联动网络,并给出股票的联动强度降序排名。我们使用军工相关股票的周度数据,构造收益矩阵
\[
R=\big[r_i(t)\big]_{t=1..T,~i=1..N}\in\mathbb{R}^{T\times N},
\]
其中 $T$ 为样本周数、$N$ 为股票数,$r_i(t)$ 表示股票 $i$ 在第 $t$ 周的收益(单位:百分比)。对同一周出现的重复记录取均值以实现时间对齐。

为保证统计稳健性,数据预处理包括:
- 分位数截断:对收益做上下各 $0.5\%$ 的分位数截断以抑制极端值;
- 限制性填补:对小缺口采用前向/后向各最多 $2$ 期的填补;
- 覆盖筛选:窗口内仅保留覆盖周数达到阈值的股票,并在样本不足时启用窗口长度的自动回退策略。

\subsection{多视角融合的联动度量}
\label{subsec:method}
我们从三个互补视角度量两只股票 $i,j$ 的联动程度:
- 价格相关性(去极值后):皮尔森相关系数
\[
\rho_{ij}=\frac{\operatorname{cov}(r_i,r_j)}{\sigma_{r_i}\sigma_{r_j}},\qquad \rho_{ij}\leftarrow \max(\rho_{ij},\,0),
\]
强调“协同联动”,将负相关截断为 $0$;
- 形态相似性(DTW):采用带 Sakoe–Chiba 约束的动态时间规整距离 $D_{ij}$,并单调映射为相似度
\[
\mathrm{Sim}^{\mathrm{DTW}}_{ij}=1-\frac{D_{ij}-D_{\min}}{D_{\max}-D_{\min}}\in[0,1];
\]
- 尾部依赖:上下尾共同极端事件比例
\[
\lambda_{ij}=\tfrac{1}{2}\!\left[\Pr\big(r_i>q_i^{0.9},\,r_j>q_j^{0.9}\big)+\Pr\big(r_i<q_i^{0.1},\,r_j<q_j^{0.1}\big)\right],
\]
其中 $q_i^{\alpha}$ 为股票 $i$ 收益的 $\alpha$ 分位数。

综合联动权重定义为
\[
w_{ij}=\alpha\,\rho_{ij}+\beta\,\mathrm{Sim}^{\mathrm{DTW}}_{ij}+\gamma\,\lambda_{ij},\qquad
\alpha=0.45,~\beta=0.35,~\gamma=0.20,
\]
并令 $W=[w_{ij}]$ 的对角元为 $0$。每只股票的净联动强度定义为 $S_i=\sum_{j}w_{ij}$,据此进行降序排名;并在 $W$ 上采用 Louvain 方法进行社区划分,作为网络结构的解释依据。

\subsection{实验设置}
\label{subsec:exp-setup}
除非另有说明,默认设置如下:
- 窗口长度:$52$ 周;自动回退序列:$104,~208,~\text{全样本}$;
- 覆盖筛选:令 $T$ 为窗口内周数,先以阈值 $\max(10,~\lfloor0.3T\rfloor)$ 初筛,再以 $\max(8,~\lfloor0.3T\rfloor)$ 复筛;
- 缺失填补:前向/后向各最多 $2$ 期;
- DTW 带宽比:$0.10$(Sakoe–Chiba 约束);相关性最小重叠样本:$8$;
- 尾部分位数:$q_u=0.90,\;q_l=0.10$;
- 社区发现:Louvain(resolution $=1.0$,随机种子 $42$)。

\subsection{结果展示}
\label{subsec:results}
图~\ref{fig:w-heatmap} 给出了按社区顺序重排后的综合联动权重矩阵 $W$ 的热力图,可见清晰的块状结构,表明板块内部存在若干相对紧密的子群。图~\ref{fig:network} 为在阈值 $w_{ij}\ge 0.7$ 下得到的联动网络图,节点大小与净联动强度 $S_i$ 成正比,颜色表示社区。图~\ref{fig:topk} 给出了综合联动系数 Top-20 的柱状图,用于直观展示板块核心股票。

% —— 采用方案B版式:上两图并排、下方整行 ——
\begin{figure}[htbp]
	\centering
	\begin{minipage}{0.49\linewidth}
		\centering
		\IncludeFigure[width=\linewidth]{D:/1.scripts/output/figs/fig_W_heatmap.png}
		\captionof{figure}{综合联动权重矩阵 $W$(按社区重排)}
		\label{fig:w-heatmap}
	\end{minipage}\hfill
	\begin{minipage}{0.49\linewidth}
		\centering
		\IncludeFigure[width=\linewidth]{D:/1.scripts/output/figs/fig_network.png}
		\captionof{figure}{联动网络(阈值 $0.7$;节点大小=净联动强度;颜色=社区)}
		\label{fig:network}
	\end{minipage}
\end{figure}

% 下方整行:Top-20 柱状图

\begin{figure}[htbp]
	\centering
	\IncludeFigure[width=0.9\linewidth]{D:/1.scripts/output/figs/fig_topk_bar.png}
	\caption{综合联动系数 Top-20 柱状图}
	\label{fig:topk}
\end{figure}


\newpage
\subsection{稳健性与基线对比}
\label{subsec:robustness}
我们比较 $52$ 周与 $104$ 周窗口下的排名一致性,报告 Spearman 秩相关与 Top-$K$($K=50$)重合率;并将融合模型($\alpha{=}0.45,\beta{=}0.35,\gamma{=}0.20$)与仅相关性基线($\alpha{=}1,\beta{=}\gamma{=}0$)对比。若对比结果已输出为文本文件 \texttt{output/figs/baseline\_compare.txt},可直接在下方展示:

\bigskip
\noindent
\IfFileExists{output/figs/baseline_compare.txt}{%
	\lstinputlisting[basicstyle=\ttfamily\small,frame=single]{output/figs/baseline_compare.txt}%
}{\fbox{\parbox[c][0.12\textheight][c]{0.9\linewidth}{\centering 缺少文件:\texttt{output/figs/baseline\_compare.txt}\\请先运行对比脚本生成结果后再编译}}}

\subsection{结论}
\label{subsec:conclusion}
本文基于价格相关性、形态相似性(DTW)与尾部依赖三种互补视角,构建军工板块的联动网络并给出净联动强度排名。热力图与阈值化网络揭示了板块内部的块状结构与“核心股”群体;与仅相关性基线相比,融合模型在结构连贯性与 Top-$K$ 稳定性方面更具优势。

\end{document}